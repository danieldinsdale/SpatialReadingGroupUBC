\documentclass{beamer}
\usepackage{subfigure}

% There are many different themes available for Beamer. A comprehensive
% list with examples is given here:
% http://deic.uab.es/~iblanes/beamer_gallery/index_by_theme.html
% You can uncomment the themes below if you would like to use a different
% one:
%\usetheme{AnnArbor}
%\usetheme{Antibes}
%\usetheme{Bergen}
%\usetheme{Berkeley}
%\usetheme{Berlin}
%\usetheme{Boadilla}
%\usetheme{boxes}
%\usetheme{CambridgeUS}
%\usetheme{Copenhagen}
%\usetheme{Darmstadt}
%\usetheme{Frankfurt}
%\usetheme{Goettingen}
%\usetheme{Hannover}
%\usetheme{Ilmenau}
%\usetheme{JuanLesPins}
%\usetheme{Luebeck}
%\usetheme{Madrid}
%\usetheme{Malmoe}
%\usetheme{Marburg}
%\usetheme{Montpellier}
%\usetheme{PaloAlto}
%\usetheme{Pittsburgh}
%\usetheme{Rochester}
%\usetheme{Singapore}
%\usetheme{Szeged}
%\usetheme{Warsaw}

\title{Spatial Reading Group}

% A subtitle is optional and this may be deleted
\subtitle{Optional Subtitle}

%\date{Conference Name, 2013}
% - Either use conference name or its abbreviation.
% - Not really informative to the audience, more for people (including
%   yourself) who are reading the slides online

\subject{blah}
% This is only inserted into the PDF information catalog. Can be left
% out. 

% If you have a file called "university-logo-filename.xxx", where xxx
% is a graphic format that can be processed by latex or pdflatex,
% resp., then you can add a logo as follows:

% \pgfdeclareimage[height=0.5cm]{university-logo}{university-logo-filename}
% \logo{\pgfuseimage{university-logo}}

% Delete this, if you do not want the table of contents to pop up at
% the beginning of each subsection:
\AtBeginSubsection[]
{
  \begin{frame}<beamer>{Outline}
    \tableofcontents[currentsection,currentsubsection]
  \end{frame}
}

% Let's get started
\begin{document}

\begin{frame}
  \titlepage
\end{frame}

\begin{frame}{Outline}
  \tableofcontents
  % You might wish to add the option [pausesections]
\end{frame}

% Section and subsections will appear in the presentation overview
% and table of contents.
\section{First Main Section}

\subsection{First Subsection}

\begin{frame}{First Slide Title}{Optional Subtitle}
  \begin{itemize}
  \item {
    My first point.
  }
  \item {
    My second point.
  }
  \end{itemize}
\end{frame}

\subsection{Second Subsection}

% You can reveal the parts of a slide one at a time
% with the \pause command:
\begin{frame}{Second Slide Title}
  \begin{itemize}
  \item {
    First item.
    \pause % The slide will pause after showing the first item
  }
  \item {   
    Second item.
  }
  % You can also specify when the content should appear
  % by using <n->:
  \item<3-> {
    Third item.
  }
  \item<4-> {
    Fourth item.
  }
  % or you can use the \uncover command to reveal general
  % content (not just \items):
  \item<5-> {
    Fifth item. \uncover<6->{Extra text in the fifth item.}
  }
  \end{itemize}
\end{frame}

\section{Extension - Preferential Sampling}
\subsection{The Problem}

\begin{frame}{Preferential Sampling}
\begin{block}{The Problem}
\begin{itemize}
\item So far we have assumed the sampling locations $X$ are fixed, or assumed known.
\item What if the sampling locations depend on the underlying field $S$?
\end{itemize}
\end{block}

\begin{example}
\begin{itemize}
\item Pollution data from measuring stations
\item Ocean temperature data from marine mammals
\item Lead concentration in Galicia (to be shown)
\end{itemize}
\end{example}
\end{frame}

\begin{frame}{Preferential Sampling}

\begin{figure}
\centering
\caption{Example of a single realisation of $S$ and corresponding 100 sampling locations selected using a spatial Poisson Process with intensity $\lambda(x)=\exp(\beta S(x))$.\label{fig:PrefSimPlot}}
\subfigure[Example of 100 preferentially sampled locations ($\beta=2$)]{\label{fig:a}\includegraphics[width=50mm]{Images/DiggleFakeSimPlot.pdf}}
\subfigure[Example of 100 non-preferentially sampled locations ($\beta=0$)]{\label{fig:b}\includegraphics[width=50mm]{Images/DiggleFakeSimPlotNP.pdf}}
\end{figure}
\end{frame}

\begin{frame}{Preferential Sampling}
\begin{block}{Solution}
\begin{itemize}
\item We must account for the dependence between $X$ and $S$.
\begin{equation}
L(\boldsymbol{\theta})=\int \left[X,Y,S\right]\mathrm{d}S.
\end{equation}
\item Diggle et al. 2010 - Monte Carlo 
\item Integrated Nested Laplace Approximation (INLA) - Joe
\item Template Model Builder - Danny
\end{itemize}
\end{block}

\end{frame}

% Placing a * after \section means it will not show in the
% outline or table of contents.
\section*{Summary}

\begin{frame}{Summary}
  \begin{itemize}
  \item
    The \alert{first main message} of your talk in one or two lines.
  \item
    The \alert{second main message} of your talk in one or two lines.
  \item
    Perhaps a \alert{third message}, but not more than that.
  \end{itemize}
  
  \begin{itemize}
  \item
    Outlook
    \begin{itemize}
    \item
      Something you haven't solved.
    \item
      Something else you haven't solved.
    \end{itemize}
  \end{itemize}
\end{frame}



% All of the following is optional and typically not needed. 
\appendix
\section<presentation>*{\appendixname}
\subsection<presentation>*{For Further Reading}

\begin{frame}[allowframebreaks]
  \frametitle<presentation>{For Further Reading}
    
  \begin{thebibliography}{10}
    
  \beamertemplatebookbibitems
  % Start with overview books.

  \bibitem{Author1990}
    A.~Author.
    \newblock {\em Handbook of Everything}.
    \newblock Some Press, 1990.
 
    
  \beamertemplatearticlebibitems
  % Followed by interesting articles. Keep the list short. 

  \bibitem{Someone2000}
    S.~Someone.
    \newblock On this and that.
    \newblock {\em Journal of This and That}, 2(1):50--100,
    2000.
  \end{thebibliography}
\end{frame}

\end{document}




